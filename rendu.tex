\documentclass{article}

\usepackage[utf8]{inputenc}

\author{Yves Sabourin, Julien Guyot}

\title{Mini ML : implémentation d'un typeur et d'un intérpreteur}

\begin{document}
\section{introduction}
Nous avons choisi le projet d'implémentation du typeur, et de
l'interprète de Mini-ML.

Nous avons, en plus des fichiers donnés, créé les fichiers .ml
suivants :
\begin{itemize}
\item[types.ml] Comme son nom l'indique, il contient la définition des
  types de MiniML. Il contient aussi une fonction ``str\_of\_t'', qui
  affiche la chaîne de caractères associée à un type donné/
\item[interpretation.ml] Contient l'interpréteur, qui a pour but
  d'interpréter des  expressions
\item[typing.ml] Contient les fonctions nécessaires au typage d'une
  expression donnée. Nous reviendrons plus tard sur son
  fonctionnement, ce typage est monomorphe.
  De plus, les erreurs de typage sont peu à pas gérées (une exception
  ``failure'' quand on voit une erreur)
\end[itemize]


\section{Typage}

Le principe du typage est le suivant :
\begin{enumerate}
\item Annotation de toutes les sous-expressions de l'expression que
  l'on veut typer par un identifiant (qui est un entier). Cela est
  géré par la méthode annotation\_pexp, prenant en paramètre
  l'expression à typer, et l'environnement dans lequel on se trouve
  
\item Écriture des contraintes : Le typage d'une expression peut être
  vu comme le respect de certaines contraintes (par exemple, si on
  voit ``a+b'', les contraintes à respecter seraient que a et b soient
  des entiers). \\
  Ainsi, la fonction ecriture\_equat ``note'' les contraintes, qui
  sont de type ``equation'', afin de connaitre les contraintes à
  respecter.
\item Résolution des équations-contraintes, par unification. On
  obtient une liste de substitutions. Cela est géré par la fonction
  unification
\item Substitution du type de l'expression qu'on cherchait à typer :
  on obtient dont le type de l'expression.
\end{enumerate}

Comme mentionné plus tôt, nous n'avons pas réussi à implémenter le
typage polymorphe. Cela est principalement dû au fait que 

\section{Interprétation}

L'interprétation fut moins difficile à implémenter


\end{document}
